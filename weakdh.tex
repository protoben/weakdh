\pdfminorversion=4
\documentclass[presentation, aspectratio=54]{beamer}

\usepackage[utf8]{inputenc}
\usepackage[T1]{fontenc}
\usepackage{fixltx2e}
\usepackage{graphicx}
\usepackage{float}
\usepackage{wrapfig}
\usepackage[normalem]{ulem}
\usepackage{amsmath}
\usepackage{textcomp}
\usepackage{amssymb}
\usepackage{hyperref}

\tolerance=1000
\input{SUTD-dark}
\usepackage{textgreek}
\usetheme{default}

\definecolor{orchid}{RGB}{153, 102, 255}
\newcommand{\cyan}[1]{\textcolor{cyan}{#1}}
\newcommand{\magenta}[1]{\textcolor{magenta}{#1}}
\newcommand{\purple}[1]{\textcolor{orchid}{#1}}

\author{Asher Toback and Ben Hamlin}
\date{May 31, 2018}
\title{Imperfect Forward Secrecy \\ by David Adrian et al.}
\begin{document}

\maketitle
\section{Background}

%%%%%%%%%%%%%%%%%%%%%%%%%%%%%%%%%%%%%%%%%%%%%%%%%%%%%%%%%%%%%%%%%%%%%%%%%%%%%%%

\begin{frame}{Diffie-Hellman}

\cyan{Alice} and \magenta{Bob} want to do some symmetric crypto. They need a
shared \purple{key}. But their communication channel is public...

\end{frame}

%%%%%%%%%%%%%%%%%%%%%%%%%%%%%%%%%%%%%%%%%%%%%%%%%%%%%%%%%%%%%%%%%%%%%%%%%%%%%%%

\begin{frame}{Diffie-Hellman}

\begin{columns}
\begin{column}{0.5\textwidth}
\cyan{Alice} has...
\begin{itemize}
\item a private exponent $\cyan{a}$
\end{itemize}
\end{column}\hfill
\begin{column}{0.5\textwidth}
\magenta{Bob} has...
\begin{itemize}
\item a private exponent $\magenta{b}$
\end{itemize}
\end{column}
\end{columns}
\vspace{20pt}
They share publicly...
\begin{itemize}
\item an $n$-bit prime modulus $p$
\item a generator $g$ of the subgroup $G \subseteq \mathbb{Z}^*_p$
\end{itemize}

\end{frame}

%%%%%%%%%%%%%%%%%%%%%%%%%%%%%%%%%%%%%%%%%%%%%%%%%%%%%%%%%%%%%%%%%%%%%%%%%%%%%%%

\begin{frame}{Diffie-Hellman}

\begin{columns}
\begin{column}{0.5\textwidth}
Alice publicly sends...
\[\cyan{A} = g^{\cyan{a}}\]
\end{column}
\begin{column}{0.5\textwidth}
Bob publicly sends...
\[\magenta{B} = g^{\magenta{b}}\]
\end{column}
\end{columns}
\vspace{20pt}
Their \purple{key} will be...
\[\purple{K}
        = \cyan{A}^{\magenta{b}}
        = \magenta{B}^{\cyan{a}}
        = g^{\cyan{a}\magenta{b}}\]

\end{frame}

%%%%%%%%%%%%%%%%%%%%%%%%%%%%%%%%%%%%%%%%%%%%%%%%%%%%%%%%%%%%%%%%%%%%%%%%%%%%%%%

\begin{frame}{Diffie-Hellman}

An eavesdropper Eve knows...
\begin{itemize}
\item the modulus $p$
\item the generator $g$
\item $\cyan{A} = g^{\cyan{a}}$
\item $\magenta{B} = g^{\magenta{b}}$
\end{itemize}
She wants $\purple{K} = g^{\cyan{a}\magenta{b}}$. What can she do?

\end{frame}

%%%%%%%%%%%%%%%%%%%%%%%%%%%%%%%%%%%%%%%%%%%%%%%%%%%%%%%%%%%%%%%%%%%%%%%%%%%%%%%

\begin{frame}{Diffie-Hellman}

Eve can calculate...
\[\cyan{a} = \log_g \cyan{A}\]
\[\purple{K} = \magenta{B}^{\cyan{a}}\]
This takes time $O(|G|^{\frac{1}{2}}) \approx O(\sqrt{2^{\log p}})$. [CITE HERE]

\end{frame}

\end{document}
