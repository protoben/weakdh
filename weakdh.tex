\pdfminorversion=4
\documentclass[presentation, aspectratio=54]{beamer}

\usepackage[utf8]{inputenc}
\usepackage[T1]{fontenc}
\usepackage{fixltx2e}
\usepackage{graphicx}
\usepackage{float}
\usepackage{wrapfig}
\usepackage[normalem]{ulem}
\usepackage{amsmath}
\usepackage{textcomp}
\usepackage{amssymb}
\usepackage{hyperref}

\tolerance=1000
% Include this file for dark background SUTD style
% Nils Ole Tippenhauer, SUTD, 2014
% This is somewhat hackish right now. Should be transformed into proper beamer theme
\mode<presentation>

\usecolortheme{rose}
\useinnertheme{rounded}
\usecolortheme{dolphin}
\useoutertheme{infolines}

% more stuff from Boadilla sty
\setbeamersize{text margin left=1em,text margin right=1em}
\setbeamertemplate{headline}[default] % disables headline
%\mode
%<all>

% disable tool bar
\beamertemplatenavigationsymbolsempty

% Logo
%\usepackage{pgf}  
%\logo{\pgfputat{\pgfxy(-1,8.32)}{\pgfbox[center,base]{\includegraphics[height=0.6cm]{./SUTD_logo_eng_hor_rev}}}}

% Fonts
\usepackage{helvet}
\usepackage{times} % for the serif stuff

% Title page, simple version
\setbeamertemplate{title page}{
%{\usebeamerfont{subtitle}\insertsubtitle\\}
 \usebeamerfont{title}\LARGE\bf\inserttitle\hrule
 \vspace{0.1cm}
 \usebeamerfont{author}\hfill\insertauthor
}
% for later: 
% \usebeamerfont{subtitle}, \insertsubtitle
% \usebeamerfont{institute}, \insertinstitute
% \usebeamercolor[fg]{titlegraphic}, \inserttitlegraphic
% \usebeamerfont{author}, \insertaddress

% hline for the titles
\setbeamertemplate{frametitle}{\usebeamerfont{title}\bf\insertframetitle\par\vskip-6pt\line(1,0){290}}

% now the colors
\definecolor{SUTDred}{RGB}{153,0,51}
\definecolor{lgrey}{RGB}{204,204,204}
\definecolor{mgrey}{RGB}{102,102,102}
\definecolor{dgrey}{RGB}{51,51,51}
\definecolor{Dgrey}{RGB}{37,37,37}%from the SUTD PowerPoint Template
\definecolor{white}{RGB}{255,255,255}

\setbeamercolor{normal text}{fg=white,bg=Dgrey}

\setbeamercolor{section number projected}{fg=white}
\setbeamercolor{section in toc}{fg=white}

\setbeamercolor{alerted text}{fg=SUTDred}

\setbeamertemplate{footline}{
   \begin{beamercolorbox}[ht=4ex,leftskip=0.2cm,rightskip=.2cm]{footer line}
    \hrule
    \vspace{0.1cm}
    \insertauthor \hspace{0.2cm} \inserttitle /\insertsection \hfill \insertdate\hspace{0.2cm}\insertframenumber\,/\,\inserttotalframenumber
    \vspace{0.1cm}
 \end{beamercolorbox}
}

\setbeamercolor*{item}{fg=white}
\useitemizeitemtemplate{\Large\raise-1.0pt\hbox{\textbullet}}
\usesubitemizeitemtemplate{%
    \tiny\raise1.5pt\hbox{$\blacktriangleright$}%
}
\usesubsubitemizeitemtemplate{\large\raise-1.0pt\hbox{\textbullet}}

\setbeamercolor*{palette primary}{fg=black,bg=white}
\setbeamercolor*{palette secondary}{fg=black,bg=mgrey}
\setbeamercolor*{palette tertiary}{fg=white,bg=black}
\setbeamercolor*{palette quaternary}{fg=white,bg=black}

\setbeamercolor{title}{fg=white}
\setbeamercolor{frametitle}{fg=white}
\setbeamercolor{framesubtitle}{fg=white}
\setbeamercolor{caption}{fg=white}

%\setbeamercolor{block title}{fg=white,bg=dgrey}
%\setbeamercolor{block body}{fg=black,bg=lgrey}
% test invisible blocks
\setbeamercolor{block title}{fg=white,bg=Dgrey}
\setbeamercolor{block body}{fg=white,bg=Dgrey}

% \setbeamercolor{block title alerted}{parent=alerted text,bg=black!15}
\setbeamercolor{block title alerted}{fg=white,bg=dgrey}
\setbeamercolor{block body alerted}{fg=black,bg=lgrey}
\setbeamercolor{block title example}{fg=white,bg=dgrey}
\setbeamercolor{block body example}{fg=black,bg=lgrey}

% enumerate
%\setbeamertemplate{enumerate items}[circle]
\setbeamercolor{enumerate items}{fg=black}
\setbeamertemplate{enumerate items}[square]
\setbeamercolor{item projected}{fg=black, bg=white}

\usepackage{textgreek}
\usetheme{default}

\definecolor{orchid}{RGB}{153, 102, 255}
\newcommand{\cyan}[1]{\textcolor{cyan}{#1}}
\newcommand{\magenta}[1]{\textcolor{magenta}{#1}}
\newcommand{\purple}[1]{\textcolor{orchid}{#1}}

\author{Asher Toback and Ben Hamlin}
\date{May 31, 2018}
\title{Imperfect Forward Secrecy \\ by David Adrian et al.}
\begin{document}

\maketitle
\section{Background}

%%%%%%%%%%%%%%%%%%%%%%%%%%%%%%%%%%%%%%%%%%%%%%%%%%%%%%%%%%%%%%%%%%%%%%%%%%%%%%%

\begin{frame}{Diffie-Hellman}

\cyan{Alice} and \magenta{Bob} want to do some symmetric crypto. They need a
shared \purple{key}. But their communication channel is public...

\end{frame}

%%%%%%%%%%%%%%%%%%%%%%%%%%%%%%%%%%%%%%%%%%%%%%%%%%%%%%%%%%%%%%%%%%%%%%%%%%%%%%%

\begin{frame}{Diffie-Hellman}

\begin{columns}
\begin{column}{0.5\textwidth}
\cyan{Alice} has...
\begin{itemize}
\item a private exponent $\cyan{a}$
\end{itemize}
\end{column}\hfill
\begin{column}{0.5\textwidth}
\magenta{Bob} has...
\begin{itemize}
\item a private exponent $\magenta{b}$
\end{itemize}
\end{column}
\end{columns}
\vspace{20pt}
They share publicly...
\begin{itemize}
\item an $n$-bit prime modulus $p$
\item a generator $g$ of the subgroup $G \subseteq \mathbb{Z}^*_p$
\end{itemize}

\end{frame}

%%%%%%%%%%%%%%%%%%%%%%%%%%%%%%%%%%%%%%%%%%%%%%%%%%%%%%%%%%%%%%%%%%%%%%%%%%%%%%%

\begin{frame}{Diffie-Hellman}

\begin{columns}
\begin{column}{0.5\textwidth}
Alice publicly sends...
\[\cyan{A} = g^{\cyan{a}}\]
\end{column}
\begin{column}{0.5\textwidth}
Bob publicly sends...
\[\magenta{B} = g^{\magenta{b}}\]
\end{column}
\end{columns}
\vspace{20pt}
Their \purple{key} will be...
\[\purple{K}
        = \cyan{A}^{\magenta{b}}
        = \magenta{B}^{\cyan{a}}
        = g^{\cyan{a}\magenta{b}}\]

\end{frame}

%%%%%%%%%%%%%%%%%%%%%%%%%%%%%%%%%%%%%%%%%%%%%%%%%%%%%%%%%%%%%%%%%%%%%%%%%%%%%%%

\begin{frame}{Diffie-Hellman}

An eavesdropper Eve knows...
\begin{itemize}
\item the modulus $p$
\item the generator $g$
\item $\cyan{A} = g^{\cyan{a}}$
\item $\magenta{B} = g^{\magenta{b}}$
\end{itemize}
She wants $\purple{K} = g^{\cyan{a}\magenta{b}}$. What can she do?

\end{frame}

%%%%%%%%%%%%%%%%%%%%%%%%%%%%%%%%%%%%%%%%%%%%%%%%%%%%%%%%%%%%%%%%%%%%%%%%%%%%%%%

\begin{frame}{Diffie-Hellman}

Eve can calculate...
\[\cyan{a} = \log_g \cyan{A}\]
\[\purple{K} = \magenta{B}^{\cyan{a}}\]
This takes time $O(|G|^{\frac{1}{2}}) \approx O(\sqrt{2^{\log p}})$. [CITE HERE]

\end{frame}

\end{document}
